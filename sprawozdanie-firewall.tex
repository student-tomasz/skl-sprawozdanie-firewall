\documentclass[a4paper,10pt,notitlepage]{article}
\usepackage{sprawozdanie-ato}


\begin{document}


\title{\
Laboratorium Sieci Komputerowych\\\
Konfiguracja zapory sieciowej
}
\author{\
Tomasz Cudziło, Barnaba Turek\\
\textsc{PW EE Informatyka}\\[6pt]
}
\date{\today}

\maketitle
\tableofcontents


\section{Cel ćwiczenia}

W ramach laboratorium wykonano konfigurację zapory sieciowej do blokady
określonych typów transmisji.

W pierwszej części blokowano ruch \ssh{} pochodzący z maszyny \volt{} na
dodatkowym interfejsie. Blokadę wykonano korzystając z \pf, następnie z \ipfw,
podając reguły w trybie bezstanowym. W drugiej części wszyscy żyli długo i
szczęśliwie, bo jeszcze nie znali zadania\dots


\section{Wykonanie ćwiczenia}

%\input{./tex/statyczna-blokada-ssh.tex}


%\section{Bibliografia}

\begin{thebibliography}{99}
\bibitem{wstep:stevens}
    \emph{TCP/IP Illustrated --- 2nd ed.}; Kevin R. Fall, W. Richard Stevens; Pearson Education, Inc.\\
    Rozdział 7.2
\bibitem{bsd:firewall}
    \emph{FreeBSD Handbook}\\
    Rozdział 31 Firewalls
% Nah mon. Cytujemy źródła Wikipedii, nie Wikipedię.
% \bibitem{wiki:firewall}
%     \emph{Wikipedia}\\
%     Firewall (computing)
\end{thebibliography}



\end{document}
