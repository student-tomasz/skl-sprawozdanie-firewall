\subsection{Teoria}

Zapory sieciowe to narzędzia software'owe lub hardware'owe służące ochronie
pewnego wydzielonego obszaru sieci. Umieszcza się je pomiędzy siecią wewnętrzną,
która ma podlegać ochronie, a siecią zewnętrzną, z której spodziewamy się ataków.

Zapory dzielimy na filtrujące pakiety (\ang{packet-filtering firewall}) i
pośredniczące (\ang{proxy-firewall}) \cite{wstep:stevens}. Ćwiczenie obejmuje
tylko zapory pierwszego typu.

Zapory filtrujące działają na zasadzie trasownika wyposażonego w reguły
filtrujące. Na podstawie tych reguł trasownik podejmuje decyzje, czy dany pakiet
powinien być przekazany dalej, czy odrzucony. Za pomocą tych reguł administrator
sieci stara się tak skonfigurować zaporę, aby uniemożliwić niepożądany ruch
sieciowy.

Blokowanie ruchu zachodzi na podstawie reguł o różnych sposobach analizy
pakietów.

Najprostszym typem reguł są \textbf{reguły bezstanowe}. W regułach tego typu
pakiety są odrzucane, bądź nie, tylko i wyłącznie na podstawie swojej
zawartości. Stosując reguły tego tego typu można zablokować na przykład
połączenia na konkretny port, lub połączenia z konkretnymi adresami. Przykład
zastosowania reguł tego typu znajduje się w sekcji \ref{sec:bezstanowe}.

Nieco bardziej złożone są reguły stanowe (\ang{stateful packet inspection}).
Zapory zbudowane w oparciu o reguły tego typu potrafią rozpoznać pakiety jako
należące do konkretnego połączenia. Połączenie rozumiane jest w sensie ogólnym
jako wymiana informacji pomiędzy dwoma komputerami, a niekoniecznie w sensie
połączenia \tcp. Dokładniejszy opis działania reguł tego typu oraz przykład
można znaleźć w sekcji \ref{sec:stanowe}.

Ponadto możliwe jest filtrowanie w oparciu o warstwę aplikacji. Reguły tego typu
mogą rozpoznać konkretne protokoły warstwy aplikacji -- BitTorrent czy SSH. Jest
to możliwe dzięki głębokiej inspekcji pakietów, niezależnie od tego czy
korzystają z domyślnego portu \cite{wstep:stevens}. Reguły tego typu nie były
przedmiotem ćwiczenia.


\subsection{Rozwiązania dostępne w systemie FreeBSD}

System \bsd{} udostępnia trzy rozwiązania realizujące funkcjonalność zapory
sieciowej: \pf{} -- PacketFilter, \ipf{} -- IPFILTER oraz \ipfw{} -- IPFIREWALL.

Wszystkie te rozwiązania to dojrzałe systemy pozwalające na dowolną konfigurację
stanowej lub bezstanowej zapory. Najważniejsze różnice pomiędzy nimi to język
stosowany do konfiguracji oraz narzędzia do kształtowania ruchu z którymi
współpracują \cite{bsd:firewalls}.
