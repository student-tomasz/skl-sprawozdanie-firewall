\subsubsection{Teoria}

Zapory sieciowe to narzędzia (budowane jako software lub hardware) służące ochronie pewnego wydzielonego obszaru sieci.
Umieszcza się je pomiędzy siecią, która ma podlegać ochronie, a siecią z której spodziewamy się ataków (najczęściej Internet).

Zapory dzielimy na filtrujące pakiety (\ang{packet-filtering firewall}) i pośredniczące(\ang{proxy-firewall})\cite{wstep:stevens}.
Ćwiczenie obejmuje tylko zapory pierwszego typu.

Zapory filtrujące działają na zasadzie trasownika wyposażonego w reguły filtrujące.
Na podstawie tych reguł trasownik podejmuje decyzje, czy dany pakiet powinien być przekazany dalej, czy odrzucony.
Za pomocą tych reguł administrator sieci stara się tak skonfigurować zaporę, aby uniemożliwić niepożądany ruch sieciowy.

Blokowanie ruchu może odbywać się na podstawie różnych rodzajów reguł.

Najprostszym typem reguł są reguły bezstanowe.
W regułach tego typu pakiety są odrzucane bądź nie tylko i wyłącznie na podstawie swojej zawartości.
Stosując reguły tego tego typu można zablokować na przykład połączenia na konkretny port, lub połączenia z konkretnymi adresami.
Przykład zastosowania reguł tego typu znajduje się w sekcji \ref{sec:bezstanowe}.

Nieco bardziej złożone są reguły stanowe(\ang{stateful packet inspection}).
Zapory zbudowane w oparciu o reguły tego typu potrafią rozpoznać pakiety jako należące do konkretnego połączenia (rozumianego w sensie ogólnym jako wymiana informacji pomiędzy dwoma komputerami, a niekoniecznie w sensie \tcp{}).
Dokładniejszy opis działania reguł tego typu oraz przykład można znaleźć w sekcji \ref{sec:stanowe}.

Ponadto możliwe jest także filtrowanie w oparciu o warstwę aplikacji.
Reguły tego typu mogą rozpoznać konkretne protokoły warstwy aplikacji(np. BitTorrent, SSH) na podstawie głębokiej inspekcji pakietów, niezależnie od tego czy korzystają z domyślnego portu\cite{wiki:firewall}.
Reguły tego typu nie były przedmiotem ćwiczenia.

\subsubsection{Rozwiązania dostępne w systemie FreeBSD}
System FreeBSD udostępnia trzy rozwiązania realizujące funkcjonalność zapory sieciowej: \pf{}(PacketFilter), \ipf{}(IPFILTER) i \ipfw{}(IPFIREWALL).

Wszystkie te rozwiązania to dojrzałe systemy pozwalające na dowolną konfigurację stanowej lub bezstanowej zapory.
Najważniejsze różnice pomiędzy nimi to język stosowany do konfiguracji oraz narzędzia do kształtowania ruchu z którymi współpracują\cite{bsd:firewalls}.
