% TODO GOOBY PLS
System FreeBSD  wyposażony jest w dojrzałe i dobrze udokumentowane narzędzia do konfiguracji zapór sieciowych.

Konfiguracja nawet prostej zapory na zasadzie białej listy w złożonym systemie może wymagać dużo pracy - należy zebrać wszystkie usługi z których korzysta system.
Takie rozwiązanie jest jednak dużo bezpieczniejsze od rozwiązania opartego o czarną listę.

Konfiguracja zapory opartej o białą listę jest tym trudniejsza, jeśli nie ma fizycznego dostępu do komputera który konfigurujemy.
Łatwo można doprowadzić komputer do stanu, w którym z różnych powodów nie można się na niego zalogować.
Zapora musi być budowana tak, aby po dodaniu każdej kolejnej reguły ciągle pozwalała na wszystkie połączenia wymagane do pracy zdalnej.
Może to wymagać dodawania reguł w innej kolejności, niż będą stosowane przez zaporę, dlatego \ipfw{} pozwala na podanie numeru dodawanej reguły.

Zapory z regułami stanowymi znacząco ułatwiają proste konfiguracje, w których można zaufać użytkownikom wewnątrz chronionego fragmentu sieci.
