System \bsd{} wyposażony jest w dojrzałe i dobrze udokumentowane narzędzia do
konfiguracji zapór sieciowych.

Konfiguracja nawet prostej zapory na zasadzie białej listy w złożonym systemie
może wymagać dużo pracy, ponieważ wymaga utworzenia listy wszystkich usług, z
których korzysta system. Jednak jest to rozwiązanie bezpieczniejsze od
rozwiązania opartego o czarną listę.

Konfiguracja zapory opartej o białą listę jest tym trudniejsza, jeśli nie ma
fizycznego dostępu do komputera który konfigurujemy. Łatwo można doprowadzić
zaporę do stanu, w którym będą odrzucane próby zalogowania się na maszynę.
Zapora musi być budowana tak, aby po dodaniu każdej kolejnej reguły ciągle
pozwalała na wszystkie połączenia wymagane do pracy zdalnej. Może to wymagać
dodawania reguł w innej kolejności, niż będą stosowane przez zaporę, dlatego
\ipfw{} pozwala na kolejkowanie reguł.

Zapory z regułami stanowymi znacząco ułatwiają proste konfiguracje, w których
można zaufać użytkownikom wewnątrz chronionego fragmentu sieci.
