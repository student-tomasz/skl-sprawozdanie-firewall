\subsubsection{Opis zadania}
Zadanie do zrealizowania w pierwszej części ćwiczenia polegało na takiej konfiguracji zapory, aby nie dopuszczała ona przychodzących połączeń \ssh{} z maszyny \volt{} na drugi interfejs konfigurowanej maszyny.
Zadanie zrealizowaliśmy w programie \ipfw{} oraz \pf.

%TODO machnij schemat pls, bo masz ładniejszy program i mniej roboty niż dia > pdf

Interfejs, którego zablokowanie było przedmiotem zadania -- \emo{} -- miał przydzielony adres \emoip{}.

% k1% ifconfig
% vr0: flags=8843<UP,BROADCAST,RUNNING,SIMPLEX,MULTICAST> metric 0 mtu 1500
%         options=82808<VLAN_MTU,WOL_UCAST,WOL_MAGIC,LINKSTATE>
%         ether 00:40:63:de:28:a9
%         inet 194.29.146.247 netmask 255.255.255.0 broadcast 194.29.146.255
%         media: Ethernet autoselect (100baseTX <full-duplex>)
%         status: active
% ipfw0: flags=8801<UP,SIMPLEX,MULTICAST> metric 0 mtu 65536
% lo0: flags=8049<UP,LOOPBACK,RUNNING,MULTICAST> metric 0 mtu 16384
%         options=3<RXCSUM,TXCSUM>
%         inet 127.0.0.1 netmask 255.0.0.0
% em0: flags=8843<UP,BROADCAST,RUNNING,SIMPLEX,MULTICAST> metric 0 mtu 1500
%         options=209b<RXCSUM,TXCSUM,VLAN_MTU,VLAN_HWTAGGING,VLAN_HWCSUM,WOL_MAGIC
% >
%         ether 00:1b:21:07:c9:8e
%         inet 10.146.226.128 netmask 255.255.0.0 broadcast 10.146.255.255
%         media: Ethernet autoselect (1000baseT <full-duplex>)
%         status: active

\subsubsection{Wykonanie - \pf{}}
\label{ss2:pf}
Sprawdzono, że połączenie ssh z maszyny \volt{} na adres \emoip{} jest możliwe.
Następnie skonfigurowano zaporę \pf{} w następujący sposób:

\begin{lstlisting}[caption={\texttt{/etc/pf.conf}},label=pfconf]
drugi_interfejs = "em0"
volt = "10.146.7.3"

block in quick on $drugi_interfejs proto {tcp udp} from $volt to any port ssh
\end{lstlisting}

Widać, że składnia pliku konfiguracyjnego \pf{} pozwala przypisywać symboliczne nazwy zmiennym, co zwiększa jego czytelność.

Całe zadanie realizuje reguła z ostatniej linii listingu \ref{pfconf}.

Ogólna składnia reguły w konfiguracji \pf{} jest następująca:

\begin{lstlisting}
Akcja [kierunek] [log] [quick] [on interface] [af] [proto protocol] \
   [from src_addr [port src_port]] [to dst_addr [port dst_port]] \
   [flags tcp_flags] [state]
\end{lstlisting}

\emph{Akcja} którą wybraliśmy to \textbf{block}.
Pakiet do którego zostanie zastosowana akcja \textbf{block} zostanie odrzucony.
Zależnie od konfiguracji zapora może także przesłać nadawcy informację o tym, że jego pakiet nie osiągnął celu.
Zależnie od protokołu, odpowiedzią będzie pakiet \texttt{TCP RST}(w przypadku połączenia \tcp{}) lub \texttt{ICMP Unreachable}(w przypadku pozostałych typów połączeń).

\emph{kierunek} \textbf{in} oznacza, że reguła dotyczy połączeń przychodzących.

Zastosowanie dyrektywy \emph{quick} oznacza, że jeżeli pakiet pasuje do danej reguły, to akcja (odrzucenie bądź przekazanie dalej) powinna być wykonana natychmiast.
W \pf{} inaczej niż w \ipfw{} pakiet może zostać dopasowany do wielu reguł.
Reguły dopasowywane są wg. ich umiejscowienia w pliku konfiguracyjnym - od góry do dołu.

Dyrektywa \emph{proto} przyjmuje dowolny protokół zadeklarowany w pliku \texttt{/etc/protocols}.

\pf{} stosuje akcję z pierwszej napotkanej reguły z dyrektywą \emph{quick}.
Jeżeli żadna regułą nie ma takiej dyrektywy, to zostanie zastosowana akcja z ostatniej dopasowanej reguły.

Następnie wybieramy interfejs, którego dotyczy reguła, protokół, adres nadawcy i odbiorcy.
Warto zwrócić uwagę na konstrukcję \texttt{\{udp tcp\}} - jest to lista.
Reguła zawierająca listę zostaje podczas wczytywania pliku konfiguracyjnego przepisana, jako wiele reguł.
W pamięci tworzona jest reguła dla każdej kombinacji elementów wszystkich list.

Zastosowano także słowo kluczowe \emph{any}, które oznacza dowolny adres docelowy.

Ostatecznie regułą określa, że blokowane mają być połączenia na port \ssh{}.
Określenie numeru portu odpowiadającego \ssh{} odbywa się za pomocą pliku \texttt{/etc/services}.

Po uruchomieniu zapory nie udało się nawiązać połączenia \ssh{} z maszyny \volt{} na \emoip{}.

\subsubsection{wykonanie - \ipfw{}}
\label{ss2:ipfw}
\ipfw{}, w przeciwieństwie do \pf{} nie posiada własnej składni plików konfiguracyjnych.
Zamiast tego reguły filtrowania konfiguruje się korzystając z polecenia \texttt{ipfw}.

Polecenie to pozwala na modyfikowanie zestawu reguł na poziomie reguły.
Pozwala to łatwo tworzyć dynamiczne zestawy reguł.
Wystarczy połączyć polecenie \ipfw{} i \cron{} aby uzyskać regułę zależną od godziny.

Rozwiązanie takie nie oznacza, że język konfiguracyjny jest ubogi w stosunku do \pf{}.
Co prawda samo polecenie \ipfw{} nie obsługuje zmiennych, ale za to możliwe jest korzystanie ze zmiennych powłoki.
Możliwe jest także używanie konstrukcji warunkowych czy pętli, co pozwala na tworzenie zaawansowanych i dynamicznych zestawów reguł bez potrzeby przeładowywania całej konfiguracji zapory.

\begin{lstlisting}
 # ipfw add 100 deny tcp from 10.146.7.3 to any dst-port 22 via em0
 # ipfw add 200 deny udp from 10.146.7.3 to any dst-port 22 via em0
\end{lstlisting}

Składnia polecenia jest dość podobna, do tego co zostało opisane w sekcji \ref{ss2:pf}.

Po dyrektywie \emph{add} podano numer reguły, która ma zostać utworzona.
Podobnie jak w \pf{} reguły dopasowane są od najniższych do najwyższych numerów.

W przeciwieństwie do \pf{} akcja jest wykonywana od razu po dopasowaniu reguły.

\subsubsection{Filtrowanie przed ustanowieniem połączenia}
Teoretycznie w protokole \tcp{} aby zablokować cały ruch wystarczy zablokować możliwość ustanowienia połączenia.

Można więc dodać do zapory następującą regułę:
\begin{lstlisting}
 # ipfw add 100 deny tcp from 10.146.7.3 to any dst-port 22 via em0 tcpflags syn
\end{lstlisting}

Reguła ta sprawi, że zostaną odrzucone wszystkie przychodzące pakiety z flagą \texttt{SYN}.

Jeżeli celem jest samo zablokowanie ruchu, to naszym zdaniem lepiej stosować rozwiązanie z sekcji \ref{ss2:ipfw}.
Reguła zastosowana w tamtej sekcji powinna działać wydajniej, ponieważ nie musi badać flag \tcp{} każdego pakietu.

Ponadto reguła z sekcji \ref{ss2:ipfw} zablokuje też już nawiązane połączenia i komunikację nie będącą częścią połączeń (błędnie wysłane pakiety, próby ataku na stos \texttt{TCP/IP} systemu za firewallem).

Stosowanie filtrowania opartego na flagach \tcp{} ma sens, jeżeli celem jest zezwolenie na połączenie pomiędzy dwoma maszynami tylko jeśli jedna z nich jest maszyną inicjującą połączenie.

Rozważmy regułę bez podanego portu:

\begin{lstlisting}
 # ipfw add 300 deny tcp from 10.146.7.3 to any via em0 tcpflags syn
\end{lstlisting}

Regułą taka pozwoli na komunikację z maszyną \volt{} za pomocą \tcp{} wtedy i tylko wtedy, kiedy maszyna volt nie jest maszyną inicjującą połączenie.
Takie rozwiązanie jest podobne do reguł stanowych, natomiast nie zużywa pamięci na przechowywanie informacji o tym, które połączenia zostały otwartę i przez którą stronę.
