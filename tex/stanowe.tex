\subsubsection{Opis zadania}
Druga część ćwiczenia polegała na zapoznaniu się z zaporą stanową.
Tę część ćwiczenia przeprowadzono wyłącznie w oparciu o program \ipfw{}.

Jako zadanie w tej części wybrano takie skonfigurowanie zapory, aby możliwe było nawiązanie połączenia ze zdalną maszyną na porcie 80.
Połączenie miało być możliwe do utworzenia tylko jeśli inicjowała je nasza maszyna.
Niemożliwe miało być natomiast nawiązanie identycznego połączenia (tj. takiego, w którym zdalna maszyna używała portu 80) przez zdalną maszynę.

Zaporę skonfigurowano na maszynie \kdwa{}.
Jako maszyny testowej, która miała się z nią łączyć użyto maszyny \kpiec{}.

% TODO - schemat z naszą maszyną i dowolną obcą

% Do tego możesz machnąć diagram podobny do diargramu sekwencji
% (trochę jak PADI-PADO w sprawozdaniu z PPP) pokazujący jak three-way
% handshake działa, kiedy k2 jest inicjatorem, a nie działa kiedy inna
% maszyna jest.

% Coś takiego:

% K2             ???
% |   -- SYN ->   |
% |   <- S+A --   |
% |   -- ACK ->   |

% K2             ???
% |   <- SYN --   |
% |   -- RST ->   |

\subsubsection{Wykonanie - \pf{}}
Wyczyszczono wszystkie reguły zapory ipfw:
\begin{lstlisting}
ipfw -q -f flush
\end{lstlisting}

Za pomocą programu \nc{} na komputerze \kpiec{} zaczęto nasłuchiwać na połączenie na porcie 80.
Sprawdzono wykonanie połączenia z komputera \kdwa{}.

Kiedy potwierdzono, że nie ma problemu z nawiązaniem takiego połączenia zbadaliśmy połączenie odwrotne.
\kdwa{} nasłuchiwało na porcie 6666, a nadawło z portu 80.
Udało się nawiązać połączenie.

% TODO Nierówne są, coś ma dodatkowy margines. Jeśli suma szerokości
% minipage'y to 1\linewidth, to zajmują więcej niż linewidth.
\begin{minipage}[b]{0.4\linewidth}
\begin{lstlisting}[caption={\kdwa{}}]
$ netcat k5 80
test
^D

$ netcat -l 6666
test test
$
\end{lstlisting}
\end{minipage}
\begin{minipage}[b]{0.12\linewidth}
  \hfill\vspace{1cm}
\end{minipage}
\begin{minipage}[b]{0.4\linewidth}
\begin{lstlisting}[caption={\kpiec{}}]
$ netcat -l 80
test


$ netcat -p 80 k2 6666
test test
^D
\end{lstlisting}
\end{minipage}

Rozpoczęto konfigurację zapory.

Dobrą praktyką jest rozpoczynanie konfiguracji zapory od zablokowania całego ruchu.
Niestety do prawidłowego działania maszyny niezbędne jest wiele połączeń: \ssh{}, \texttt{LDAP}, \texttt{NAS}.
Zdecydowaliśmy się więc na tzw. \emph{blacklisting}.

Praktyka ta polega na tym, że domyślnie wszystkie połączenia są przyjmowane.
Część połączeń, zdefiniowanych na ``czarnej liście'' jest odrzucana.
Nie jest to zalecana praktyka, ze względu na to że łatwo można przeoczyć jakieś połączenie, które powinno być zablokowane i narazić sieć na atak lub nadużycia\cite{bsd:firewall}.

W przypadku zapory stanowej można bardzo łatwo wprowadzić politykę odwrotną, tzw. politykę białej listy.
Domyślnie wszystkie pakiety przychodzące byłyby odrzucane, natomiast pakiety wychodzące tworzyłyby nowe dynamiczne reguły.
Reguły te pozwalałyby odbiorcom pakietów odpowiedzieć.

Taka prosta konfiguracja zakłada, że użytkownicy w sieci chronionej firewallem mogą być obdarzeni pełnym zaufaniem i nieograniczonym dostępem do Internetu.

\subsubsection{Konfiguracja zapory}
Na maszynie \kdwa{} zostały skonfigurowane następujące reguły zapory:

\begin{lstlisting}
00100 check-state
00150 allow tcp from any to any dst-port 80 out keep-state
00250 deny log logamount 100 ip from any to any src-port 80 in
65535 allow ip from any to any
\end{lstlisting}

Reguły \texttt{150} i \texttt{250} są podobne do tych omawianych w sekcji \ref{sec:bezstanowe}.
Zezwolono na połączenia \tcp{} wychodzące, których celem jest dowolny adres i port 80.
Zabroniono natomiast wszystkich połączeń przychodzących, których źródłem jest port 80.

Dodatkowo dyrektywa \textbf{keep-state} w regule \texttt{150} instruuje zaporę, aby dodała nowe reguły dynamiczne odpowiadające adresom i portom wykorzystanym do połączenia \cite{bsd:firewall}.
Dynamiczna reguła zostanie utworzona tylko kiedy regułą zostanie dopasowana do połączenia.

Gdyby były to jedyne reguły, nawiązanie połączenia \tcp{} przez maszynę \kdwa{} byłoby niemożliwe.

Dzieje się tak ponieważ samo utworzenie dynamicznych reguł nie wystarczy.
Należy jeszcze dodać dyrektywę, która instruuje zaporę by sprawdziła czy aktualny pakiet pasuje do którejś z dynamicznych reguł.

Zależało nam, aby dynamiczne reguły były sprawdzone zanim odrzucimy pakiet (regułą \texttt{250}).
Dodano więc dyrektywę \textbf{check-state} jako regułę \texttt{100} - pierwszą w zestawie reguł naszej zapory.

W takiej konfiguracji zapora najpierw sprawdzi, czy przychodzący pakiet pasuje do reguł dynamicznych.
Jeżeli tak, zostanie on zaakceptowany.
Jeżeli nie, zapora będzie kontynuwać działanie próbując dopasować pakiet do reguł statycznych.jo

\subsubsection{Testy}
Przeprowadzono testy analogiczne do tych na początku ćwiczenia.

%TODO Tu też
\begin{minipage}[b]{0.4\linewidth}
\begin{lstlisting}[caption={\kdwa{}}]
$ netcat k5 80
test
^D

$ netcat -l 6666
$
\end{lstlisting}
\end{minipage}
\begin{minipage}[b]{0.12\linewidth}
  \hfill\vspace{1cm}
\end{minipage}
\begin{minipage}[b]{0.4\linewidth}
\begin{lstlisting}[caption={\kpiec{}}]
$ netcat -l 80
test
$ netcat -p 80 k2 6666
test test
TEST
^D
\end{lstlisting}
\end{minipage}

Udało się osiągnąć zamierzony efekt.
W drugim przypadku testowym nie udało się nawiązać połączenia i przesłać danych do maszyny \kdwa{}.
